\documentclass{article}

% Language setting
% Replace `english' with e.g. `spanish' to change the document language
\usepackage[english]{babel}

% Set page size and margins
% Replace `letterpaper' with`a4paper' for UK/EU standard size
\usepackage[letterpaper,top=2cm,bottom=2cm,left=3cm,right=3cm,marginparwidth=1.75cm]{geometry}

% Useful packages
\usepackage{amsmath}
\usepackage{graphicx}
\usepackage[colorlinks=true, allcolors=blue]{hyperref}

\title{Combinatorical analysis of burst failures for large-scale cluster}
\author{Meng Wang}

\begin{document}
\maketitle

\section{Setup}
We consider a storage system of N drives such that $N = X \cdot Y \cdot Z$, where there are $X$ racks in the system, each rack contains $Y$ enclosures, and each enclosure contains $Z$ drives.

Let $M = Y \cdot Z$, so $M$ denotes the number of drives per rack.

In particular, we are considering ORNL Alpine system, which is composed of 39 racks. 38 racks have 8 enclosures each, and 1 rack has 4 enclosures. Each enclosure has 106 drives.

For simplicity, we assume the system contains 40 racks. Each rack contains 8 enclosures. Each enclosure contains 100 drives.

\section{Total instances with fixed number of affected racks}

Consider $f$ failures happen in $r$ racks.

We first choose $r$ racks from all the $X$ racks, which has $C_{X}^{r}$ combinations.

For each rack combination, without loss of generality, let's assume we picked racks $0, 1, 2, ..., r-1$.
We consider all the possible failure\_per\_rack cases:
\begin{eqnarray}
S(f,r,M) = \{(f_0, f_1, ..., f_{r-1}) | \sum_{\substack{0 \leq i \leq r-1 \\ 1 \leq f_i \leq M}}f_i=f\}.
\end{eqnarray}


For each $(f_0, f_1, ..., f_{r-1})$, there are $\prod_{i=0}^{r-1}C_{M}^{f_i}$ instances.

Therefore, the total number of instances is:
\begin{eqnarray}
  C_{X}^{r} \cdot \sum_{(f_0, f_1, ..., f_{r-1}) \in S} \prod_{i=0}^{r-1}C_{M}^{f_i}
\label{eq:tcf}
\end{eqnarray}

One key here is how to get $S(f,r,M)$. Note that we have the following recurrence relation:

\begin{eqnarray}
\begin{aligned}
  S(f,r,M)
  &= \{(f_0, f_1, ..., f_{r-1}) | \sum_{\substack{0 \leq i \leq r-1 \\ 0 \leq f_i \leq M}}f_i=f\} \\
    &= \bigcup_{\substack{0 \leq a \leq M \\ a \leq f-r+1}} \{(f_0, f_1, ..., f_{r-2}, a) | \sum_{\substack{0 \leq i \leq r-2 \\ 0 \leq f_i \leq M}}f_i=f-a\} \\
    &= \bigcup_{\substack{0 \leq a \leq M \\ a \leq f-r+1}} \{s \cup a | s \in S(f-a, r-1, M) \}
\end{aligned}
\label{eq:total_bt}
\end{eqnarray}



By using formula \ref{eq:total_bt}, we can get $S(f,r,M)$ using backtracking algorithm or dynamic programming. Here is an implementation using backtracking algorithm: \url{https://github.com/ucare-uchicago/mlec-sim/blob/main/src/theory/burst_theory.py#L14}

\section{Survival instances under local clustered (RAID)}\label{sec-raid}

Consider $k_l+p_l$ local-only SLEC. For easier deployment we assume $n_l=k_l+p_l$ is divisible by Z.

$n_l$ drives in the same enclosure compose a RAID disk group. Therefore a rack contains $g_l = M/n_l$ RAID groups.

For each $(f_0, f_1, ..., f_{r-1}) \in S(f,r,M)$, rack $i$ contains $fi\geq1$ failures. We need to compute for each $f_i$, how many instances can survive the $f_i$ failures in the rack.

Denote $\eta(f_i, g_l)$ as the number of survival instances in a rack when there are $f_i$ failures in a rack containing $g_l$ $k_l+p_l$ RAID groups.

We have the following recurrence relation (which is derived by considering what will happen if disk group 0 contains $a$ failures):

\begin{eqnarray}
\begin{aligned}
  \eta(f_i, g_l) &= \sum_{\substack{0 \leq a \leq p_l \\ a\leq f_i}} \eta(f_i-a, g_l-1) \cdot C(n_l, a)
\end{aligned}
\label{eq:raid:1}
\end{eqnarray}

We can then compute $\eta(f_i, g_l)$ based on recurrence relation \ref{eq:raid:1} and backtracking algorithm. Here is an implementation: \url{https://github.com/ucare-uchicago/mlec-sim/blob/main/src/theory/burst_theory.py#L69}

Therefore, the total survival instances of $(f_0, f_1, ..., f_{r-1})$ is $\prod_{i=0}^{r-1} \eta(f_i, g_l)$.

Therefore, the total number of survival instances in the whole system is:

\begin{eqnarray}
C_{X}^{r} \cdot \sum_{(f_0, f_1, ..., f_{r-1}) \in S} \prod_{i=0}^{r-1} \eta(f_i, g_l)
\label{eq:raid:2}
\end{eqnarray}

Therefore, the probability of data loss under $f$ failures on $r$ racks for RAID is:

\begin{eqnarray}
\begin{aligned}
\text{RAID data loss} &= \frac{C_{X}^{r} \cdot \sum_{(f_0, f_1, ..., f_{r-1}) \in S} \prod_{i=0}^{r-1} \eta(f_i, g_l)} 
{C_{X}^{r} \cdot \sum_{(f_0, f_1, ..., f_{r-1}) \in S} \prod_{i=0}^{r-1}C_{M}^{f_i}}\\
&= \frac{ \sum_{(f_0, f_1, ..., f_{r-1}) \in S} \prod_{i=0}^{r-1} \eta(f_i, g_l)} 
{\sum_{(f_0, f_1, ..., f_{r-1}) \in S} \prod_{i=0}^{r-1}C_{M}^{f_i}}
\end{aligned}
\label{eq:raid:3}
\end{eqnarray}



\section{Survival instances under local declustered parity}

It's similar to local clustered erasure in \ref{sec-raid}, but now the size of the disk group is usually larger than $n_l$. 

Suppose the size of the disk group is $D$, usually $n_l \leq D \leq Z$, where $Z$ is the size of the enclosure.

If any disk group has more than $p_l$ disk failures, then there is data loss.

So now a rack contains $g_l = M/D$ RAID groups.


Denote $\eta(f_i, g_l)$ as the number of survival instances in a rack when there are $f_i$ failures in a rack containing $g_l$ disks groups, each group contains $D$ disks and do $k_l+p_l$ declustered erasure.

We have the following recurrence relation:

\begin{eqnarray}
\begin{aligned}
  \eta(f_i, g_l) &= \sum_{\substack{0 \leq a \leq p_l \\ a\leq f_i}} \eta(f_i-a, g_l-1) \cdot C(D, a)
\end{aligned}
\label{eq:dp:1}
\end{eqnarray}


Therefore, the total number of survival instances in the whole system is:

\begin{eqnarray}
C_{X}^{r} \cdot \sum_{(f_0, f_1, ..., f_{r-1}) \in S} \prod_{i=0}^{r-1} \eta(f_i, g_l)
\label{eq:dp:2}
\end{eqnarray}

Therefore, the probability of data loss under $f$ failures on $r$ racks for RAID is:

\begin{eqnarray}
\begin{aligned}
\text{Declustered data loss} &= \frac{C_{X}^{r} \cdot \sum_{(f_0, f_1, ..., f_{r-1}) \in S} \prod_{i=0}^{r-1} \eta(f_i, g_l)} 
{C_{X}^{r} \cdot \sum_{(f_0, f_1, ..., f_{r-1}) \in S} \prod_{i=0}^{r-1}C_{M}^{f_i}}\\
&= \frac{ \sum_{(f_0, f_1, ..., f_{r-1}) \in S} \prod_{i=0}^{r-1} \eta(f_i, g_l)} 
{\sum_{(f_0, f_1, ..., f_{r-1}) \in S} \prod_{i=0}^{r-1}C_{M}^{f_i}}
\end{aligned}
\label{eq:dp:3}
\end{eqnarray}


\section{Survival instances under network clustered erasure}

TBD, this is more challenging.

\section{Survival instances under network declustered erasure}

This is easy.

Consider $n_n=k_n+p_n$ network-only declustered erasure.

There is data loss whenever there are more than $p_n$ affected racks.

Therefore:

\begin{eqnarray}
  \text{Net-declus Data loss} =
    \begin{cases}
      1 & \text{if $r > p_n$}\\
      0 & \text{otherwise}
    \end{cases}       
\label{eq:net_dp:1}
\end{eqnarray}

where $r$ is the number of affected racks.

\section{Survival instances under MLEC clustered}

TBD

\section{Survival instances under MLEC Declustered}

TBD

\end{document}
