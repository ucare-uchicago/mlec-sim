\documentclass[journal]{IEEEtran}

%\documentclass[10pt]{beamer}
%\usepackage[left=1in,top=1in,right=1in,bottom=1in]{geometry}
\newcommand*{\authorfont}{\fontfamily{phv}\selectfont}
\usepackage{lmodern}

\usepackage{animate}
\usepackage[T1]{fontenc}
\usepackage{blindtext}
\usepackage{graphicx}
\usepackage{booktabs} % For formal tables
\usepackage{amsbsy}
\usepackage[ruled]{algorithm2e} % For algorithms
\usepackage{multirow}
\usepackage{amssymb}
\usepackage{amsmath}
\usepackage{tikz}
\usepackage[utf8]{inputenc}
\usepackage[english]{babel}
\usepackage{color}
\usepackage{soul}
\usepackage{fancyhdr}
\usepackage{float}
\usepackage{hyperref} 

\usepackage{amsthm}
\usepackage{mathtools}
\usepackage{xcolor}
\usepackage{pgfplots}
\usepackage{pdfpages}

\usepackage{verbatim}
\DeclarePairedDelimiter{\ceil}{\lceil}{\rceil}
\makeatletter
\def\thm@space@setup{%
  \thm@preskip=8pt plus 2pt minus 4pt
  \thm@postskip=\thm@preskip
}
\makeatother


%\newtheorem{theorem}{Theorem}[section]
%\newtheorem{corollary}{Corollary}[theorem]
%\newtheorem{lemma}[theorem]{Lemma}
%\newtheorem{definition}{Definition}[section]
\usetikzlibrary{automata, positioning}
\renewcommand{\algorithmcfname}{ALGORITHM}
\providecommand{\tightlist}{%
  \setlength{\itemsep}{0pt}\setlength{\parskip}{0pt}}

\fancypagestyle{firstpage}{% Page style for first page
  \fancyhf{}% Clear header/footer
  \renewcommand{\headrulewidth}{0.0pt}% Header rule
  \renewcommand{\footrulewidth}{0.0pt}% Footer rule
  \fancyhead[R]{\footnotesize{\thepage}}
  % Header
  %\fancyfoot[C]{-~\thepage~-}% Footer
}


\usepackage{setspace}


% set default figure placement to htbp

\usepackage{hyperref}
\usepackage{amsmath,tabu}
\usepackage{type1cm}

% add some other packages ----------

% \usepackage{multicol}
% This should regulate where figures float
% See: https://tex.stackexchange.com/questions/2275/keeping-tables-figures-close-to-where-they-are-mentioned
\usepackage[section]{placeins}



\begin{document}
\onecolumn
% \pagenumbering{arabic}% resets `page` counter to 1 
%
% \maketitle

% \textbf{----------------------------------------}

% \textbf{Network SLEC DP Definitions}

% Assume that we have $r$ racks, and each rack has $v$ disks. We still have $n=k+m$ where k is the data chunks and m is the parity chunks. To differentiate the difference between network speed and local disk speed, we introduce $S_{net_R}$ and $S_{net_W}$ as the network read write speed.

% We also assume that there is only single enclosure per rack, as there is no implementation that differentiate rack and enclosure in \verb|mlec-sim|. Therefore, Network SLEC DP ensures that each rack will contain at most 1 chunk out of a $n=k+m$ stripe.\\


% \textbf{Repair Rate of single damaged stripes in Network SLEC DP}

% When there is a single drive failure, we know that for all the impacted stripes, the other $n-1$ chunks are going to reside on other racks. Therefore, the number of disks that might contain stripes affected by this single failure is $(r-1)*v$, and these disks are going to participate in the repair of the failed disk.

% Therefore we have the following conditions
% \begin{itemize}
%     \item We have declustered parallelism of $(r-1)*v$
%     \item We read $\mathcal{CAP}(n-1)$ amount of data
%     \item We write $\mathcal{CAP}*1$ amount of data
% \end{itemize}

% We can then calculate the duration as the sum of duration to read and write as following. We assume that $S_{net_W}=S_{net_R}=S_{net}$
% \begin{equation*}
%     Duration=\frac{(n-1)\mathcal{CAP}}{(r-1)*v*S_{net}}+\frac{\mathcal{CAP}}{(r-1)*v*S_{net}}=\frac{n\mathcal{CAP}}{(r-1)*v*S_{net}}
% \end{equation*}

% Comparing with the standard repair rate of RAID which is $\mu=\frac{\mathcal{CAP}}{S_{disk}}$, we can get
% \begin{equation*}
%     \mu_1=\mu\frac{nS_{disk}}{(r-1)*v*S_{net}}
% \end{equation*}

\textbf{----------------------------------------}

\textbf{Priority Percent Calculation Notes}


\section{When there is a single failure [distinct rack], this means that}
\begin{itemize}
  \item All the affected stripes must not have another drive on the same rack
  \item The remaining good drives to select affected stripes out of are $(r-1)B$
  \item The number of failed drive is 1 and has priority of 1
  \item The priority percent equation then becomes
\end{itemize}

\begin{equation*}
  \frac{ncr((r-1)B, n-1)*ncr(1-1, 1-1)}{ncr((r-1)B+1-1, n-1)}=\frac{ncr((r-1)B, n-1)}{ncr((r-1)B, n-1)}=1
\end{equation*}

We can see that in terms of parallel repair of DP, we can only read/write from and to the disks that do not reside on the same rack. Therefore we will have the parallelism as $(r-1)B$. 

The amplification is reading from $k$ chunks and writing to $f$ chunk.

Therefore the time needed for repair calculation is as follows
\begin{equation*}
  \frac{\mathcal{CAP}(k+f)}{S_{net}(r-1)B/f}
\end{equation*}

\textbf{When there are two failures [same rack], this means that the \textit{priority 1 stripe} works as follows}
\begin{itemize}
  \item All the affected stripes must not have another drive on the same rack
  \item The remaining good drives to select affected stripes out of are \textbf{still} $(r-1)B$ because damaged stripes will have other chunks sitting on other racks, and the damaged chunk be sitting on either one of the failed drive
  \item The number of failed drive is 2 and both still has priority of 1
  \item The priority percent equation then becomes
\end{itemize}

\begin{equation*}
  \frac{ncr((r-1)B, n-1)*ncr(2-1, 1-1)}{ncr((r-1)B+2-1, n-1)}=\frac{(r-1)B-n+2}{(r-1)B+1}
\end{equation*}

Making an example with 10 racks, 10 drives per rack, and (8+2) config, and there are two failures on the same rack, the priority percent would be
\begin{equation*}
  \frac{(10-1)*10-10+2}{(10-1)10+1}=\frac{82}{91}\approx 0.901
\end{equation*}

\textbf{When there are two failures [same rack], this means that \textit{priority 2 stripe} works as follows}
\begin{itemize}
  \item Basically same as two failures same rack, priority 1 stripe, except priority
  \item The remaining good drive is $(r-1)B$
  \item The number of failed drive is 2
  \item The priority percent equation then becomes
\end{itemize}

\begin{equation*}
  \frac{ncr((r-1)B, n-2)*ncr(2-1, 2-1)}{ncr((r-1)B+2-1, n-1)}=\frac{n-1}{(r-1)B+1}
\end{equation*}

Using the same $r=10, B=10, n=10$ example, we have priority percent equals $\frac{9}{91}\approx 0.0989$


\textbf{When there are two failures [distinct rack], this means that \textit{priority 1 stripe} works a follows}
\begin{itemize}
  \item The remaining good drive is still $(r-1)B-1$ because the priority 1 stripe will have all surviving chunks in all racks except the one that contains the failed chunk. The minus 1 is because in one of the rack containing one of the surviving chunk, there is one failed disk that happens to not impact this stripe.
  \item The number of failed drive is 2
  \item The priority percent equation then becomes
\end{itemize}

\begin{equation*}
  \frac{ncr((r-1)B-1, n-1)*ncr(2-1, 1-1)}{ncr((r-1)B-1+2-1, n-1)}=\frac{(r-1)B-n+1}{(r-1)B}
\end{equation*}

Using the same $r=10, B=10, n=10$ example, we have priority percent equals $\frac{81}{90}=0.9$

\textbf{When there are two failures [distinct rack], this means that \textit{priority 2 stripe} works a follows}
\begin{itemize}
  \item All the stripes with priority 2 have both of the chunks sitting on each of the failed drive residing in two racks. This means that the remaining good drives to select from is $(r-2)B$.
  \item The number of failed disk is 2
  \item The priority of the stripes is 2
  \item The priority percent equation then becomes
\end{itemize}

\begin{equation*}
  \frac{ncr((r-2)B, n-2)*ncr(2-1, 2-1)}{ncr((r-2)B+2-1, n-1)}=\frac{n-1}{(r-2)B+1}
\end{equation*}

\textbf{When there are three failures [distinct rack], this means that \textit{pirority 3 stripe} works as follows}
\begin{itemize}
  \item The remaining good drive is $(r-3)B$
  \item The number of failed disk is 3
  \item The priority of the stripe is 3
  \item The priority perent equation then becomes
\end{itemize}

\begin{equation*}
  \frac{ncr((r-3)B, n-3)*ncr(3-1, 3-1)}{ncr((r-3)B+3-1, n-1)}=\frac{(n-1)(n-2)}{[(r-3)B+1][(r-3)B+2]}
\end{equation*}

\textbf{When there are four failures [distinct rack], this means that \textit{priority 4 stripe} works as follows}
\begin{itemize}
  \item The remaining good drive is $(r-4)B$
  \item The number of failed disk is 4
  \item The priority of the stripe is 4
  \item The priority percent equation then becomes
\end{itemize}

\begin{equation*}
  \frac{ncr((r-4)B, n-4)*ncr(4-1, 4-1)}{ncr((r-4)B+4-1, n-1)}=\frac{(n-1)(n-2)(n-3)}{[(r-4)B+1][(r-4)B+2][(r-4)B+3]}
\end{equation*}

\textbf{When there are n failures across n distinct racks, the stripes with priority n}

The priority percent calculation should be the following. First we let the number of failures be $f$
\begin{equation*}
  \prod_{i=1}^{f-1}\frac{(n-i)}{[(r-f)B+i]}
\end{equation*}

\newpage
\singlespacing 



\end{document}